\documentclass{beamer} 
\usetheme{passau}      

%% for handouts!
% \usepackage{pgfpages}
% \pgfpagesuselayout{4 on 1}[a4paper, landscape, border shrink=5mm]
% \pgfpageslogicalpageoptions{1}{border code=\pgfusepath{stroke}}
% \pgfpageslogicalpageoptions{2}{border code=\pgfusepath{stroke}}
% \pgfpageslogicalpageoptions{3}{border code=\pgfusepath{stroke}}
% \pgfpageslogicalpageoptions{4}{border code=\pgfusepath{stroke}}

% \setbeameroption{show notes} 
%  \hypersetup{pdfpagemode=FullScreen}
  
\usepackage[utf8]{inputenc}
\usepackage[T1]{fontenc}
\usepackage[english]{babel}
\usepackage{graphicx}
\usepackage{xcolor}
\usepackage{fancyvrb}
\usepackage{fixltx2e} % fix textsub- and -superscript
\usepackage{listings}
\usepackage{paralist} % for inline lists
\usepackage{xspace}
\usepackage[compatibility=false]{caption}
\usepackage{subcaption}
\usepackage{lmodern}
\usepackage{amsmath}
\usepackage{ulem} % for line-through
   
\DeclareGraphicsExtensions{.png,.jpg,.pdf,.mps,.gif,.eps}


% fixing commands (sub- and super-scripts)

\newcommand\super[1]{\textsuperscript{#1}}
\newcommand\sub[1]{\textsubscript{#1}}
\newcommand\supersub[2]{\rlap{\super{#1}}\sub{#2}}
\newcommand{\etalchar}[1]{$^{#1}$}

% own commands

% todo
\newcommand{\todo}[1]{{\color{red}{\textbf{\textit{[#1]}}}}}
\newcommand{\TODO}[1]{\todo{#1}}
\newcommand{\todots}{\todo{\ldots}}
\frenchspacing

% typesetting
\newcommand\code[1]{\texttt{#1}}
\newcommand\feature[1]{\texttt{#1}}
\newcommand\tool[1]{\textsc{#1}}
\newcommand\metric[1]{{#1}}

%instantiations
\newcommand\ifdeff[1]{\code{\##1}\xspace}
\newcommand\ifdef[0]{{\upshape\ifdeff{ifdef}}\xspace}
\newcommand\ifdefs[0]{\ifdef{}s\xspace}

\newcommand\cppstats[0]{\tool{cppstats}\xspace}
\newcommand\cpp{\tool{cpp}\xspace}


% listings

\lstset{
  basicstyle=\ttfamily\scriptsize,
  keywordstyle=\ttfamily\scriptsize\bfseries,
  commentstyle=\ttfamily\scriptsize\itshape,
  escapechar=@,
  tabsize=2,
  numbers=left,
  firstline=1,
  rulesepcolor=\color{black},
  backgroundcolor=\color{white},
  numbersep=0.65em,
  xleftmargin=0.5em,
%  xrightmargin=1em,
  frame=top|bottom|left|right,
  showstringspaces=false,
  mathescape=true,
  columns=fullflexible,
%  escapeinside={(*@}{@*)},
  otherkeywords={endif, ifdef, elif, else, define, ifndef, defined, \#}
}


%%%%%%%%%%%%%%%%%%%%%%%%%%%%%%%%%%%%%%%%%%%%%%%%%%%%%%%%%%%%%%%%%%%%%%%%%%%%%%%%%%%%%%%
% beamer options %%%%%%%%%%%%%%%%%%%%%%%%%%%%%%%%%%%%%%%%%%%%%%%%%%%%%%%%%%%%%%%%%%%%%%
%%%%%%%%%%%%%%%%%%%%%%%%%%%%%%%%%%%%%%%%%%%%%%%%%%%%%%%%%%%%%%%%%%%%%%%%%%%%%%%%%%%%%%%
                
\beamertemplatenavigationsymbolsempty
\setbeamercovered{dynamic}

%%%%%%%%%%%%%%%%%%%%%%%%%%%%%%%%%%%%%%%%%%%%%%%%%%%%%%%%%%%%%%%%%%%%%%%%%%%%%%%%%%%%%%%
% title page definition %%%%%%%%%%%%%%%%%%%%%%%%%%%%%%%%%%%%%%%%%%%%%%%%%%%%%%%%%%%%%%%
%%%%%%%%%%%%%%%%%%%%%%%%%%%%%%%%%%%%%%%%%%%%%%%%%%%%%%%%%%%%%%%%%%%%%%%%%%%%%%%%%%%%%%%

\title[Handling \ifdef Expressions in \cppstats]{Handling \ifdef Expressions\\in \cppstats}   
% \subtitle{Untertitel} 
\author[Claus Hunsen]{Claus Hunsen}
%%\author{F.~Author\inst{1} \and S.~Another\inst{2}}
\institute{University of Passau}
% % \institute[Universities of] 
% % { 
% % \inst{1}%
% % Faculty of Informatics and Mathematics\\
% % University of Passau
% % \and
% % \inst{2}%
% % Department of Theoretical Philosophy\\
% % Univ of E}
\titlegraphic{\includegraphics[height=1cm]{unilogo.pdf} \hspace*{10pt} \includegraphics[height=1cm]{SPLGroup.pdf}}
\date[September 2014]{September 2014}

% - Use the \inst{?} command only if the authors have different
%   affiliation.  
% \author{\inst{1}} 

% - Use the \inst command only if there are several affiliations.
% - Keep it simple, no one is interested in your street address.


% Delete this, if you do not want the table of contents to pop up at
% the beginning of each subsection:
% \AtBeginSection[]
% {
% 	\begin{frame}<beamer>
% 	\frametitle{Outline}
% 		\tableofcontents[currentsection,currentsubsection]
% 	\end{frame}
% } 

% If you wish to uncover everything in a step-wise fashion, uncomment
% the following command:

%\beamerdefaultoverlayspecification{<+->}
 
\definecolor{pb}{RGB}{108,166,205}
\definecolor{ftb}{RGB}{238,92,66}
\definecolor{fam}{RGB}{102,205,0}
 
\newcommand{\coloredSquare}[1]{\color{#1}\ensuremath{\blacksquare}}
 
\begin{document}
   
% --------------------------------------------------- Slide --
\begin{frame}[plain]
  \titlepage
\end{frame} 

\begin{frame}
  \frametitle{Outline} %
  \tableofcontents  
\end{frame} 


\section[Introduction]{Introduction}
\stepcounter{subsection}


\begin{frame}
\frametitle{Outline of the \cppstats Mechanisms}

\begin{itemize}
\item Source-Code Preparation Before Generating \tool{srcML} files
	\begin{itemize}
	\item Multi-Line \ifdef Expressions
	\item Rewriting of \ifdeff{ifdef} and \ifdeff{ifndef}
	\item Removal of Include Guards
	\end{itemize}
\item Processing of Expressions During File Analysis \textcolor{colorPassauAlert}{*}
\item Global \ifdef Expression Pool
	\begin{itemize}
	\item Listing of \ifdef Expressions per File \textcolor{colorPassauAlert}{*}
	\item Construction of Global Expression Pool
	\end{itemize}
\end{itemize}

\vfill

\textcolor{colorPassauAlert}{*} -- affects scattering and tangling measurement over \ifdefs

\end{frame}


\begin{frame}
\frametitle{Studies using \cppstats}

This algorithm have been used in the following studies:

\nocite{*}
\bibliographystyle{alpha}
\begin{thebibliography}{LAL{\etalchar{+}}10}
\footnotesize

\bibitem[LAL{\etalchar{+}}10]{LiebigALKS10}
J{\"o}rg Liebig, Sven Apel, Christian Lengauer, Christian K{\"a}stner, and
  Michael Schulze. 
{\itshape An Analysis of the Variability in Forty Preprocessor-Based Software
  Product Lines}.
In {\em ICSE}, pages
  105--114. ACM, 2010.

\bibitem[LKA11]{LiebigKA11}
J{\"o}rg Liebig, Christian K{\"a}stner, and Sven Apel.
{\itshape Analyzing the Discipline of Preprocessor Annotations in 30 Million
  Lines of {C} Code}.
In {\em AOSD}, pages 191--202. ACM, 2011.

\bibitem[HZS{\etalchar{+}}14]{HunsenZSKLBA}
Claus Hunsen, Bo~Zhang, Janet Siegmund, Christian K{\"a}stner, Olaf
  Le{\ss}enich, Martin Becker, and Sven Apel.
{\itshape Preprocessor-Based Variability in Open-Source and Industrial
  Software Systems: An Empirical Study}.
{\em {Empirical Software Engineering}}, 2014.
Submitted.

\end{thebibliography}

\end{frame}


\AtBeginSection[] {
    \begin{frame}<beamer>
    \frametitle{Outline} %
    \tableofcontents[currentsection]  
    \end{frame}
}

\section{Example}
\stepcounter{subsection}

  
\begin{frame}[fragile]
\frametitle{Example of Three Files}


\begin{figure}[ht]
        \centering
        \small
        \begin{subfigure}[b]{0.3\textwidth}
					\begin{lstlisting}[language=C]
#ifdef A
  #ifdef B
  #endif
#endif
#ifdef A
#endif
					\end{lstlisting}
					\caption{Nested \ifdef in file\,\code{X.c}.\vspace{1.15em}}
					\label{fig:examples:a}
        \end{subfigure}
        \hfill
        \begin{subfigure}[b]{0.3\textwidth}
					\begin{lstlisting}[language=C, firstnumber=5]
#if  defined(A) \
     && defined(B)
#else 
#endif
					\end{lstlisting}
					\caption{\ifdeff{else} branch in file\,\code{Y.c}.\vspace{1.15em}}
					\label{fig:examples:b}
        \end{subfigure}
				\hfill
        \begin{subfigure}[b]{0.3\textwidth}
					\begin{lstlisting}[language=C, firstnumber=8]
#ifndef Z_H
#define Z_H
#ifdef C
#elif defined(D)
#endif
#endif // Z_H
					\end{lstlisting}
					\caption{\ifdeff{elif} branch and include~guard in file\,\code{Z.h}.}
					\label{fig:examples:c}
        \end{subfigure}     
        
        \caption{Short examples of the patterns that occur while using \cpp and that are treated by \cppstats. 
        	Each example and their rewriting rules are explained in this very document.}
        \label{fig:examples}
\end{figure}


\end{frame}


\section[Preparation]{Source-Code Preparation Before Generating \tool{srcML} files}
\stepcounter{subsection}

\begin{frame}[fragile]
\frametitle{Multi-Line \ifdef Expressions}


\begin{figure}[ht]
  \centering
  \begin{subfigure}[b]{0.45\textwidth}
		\begin{lstlisting}[language=C, firstnumber=5]
#if defined(A) \
		 && defined(B)
#else 
#endif
		\end{lstlisting}
		\caption{Repetition of Fig.\ \ref{fig:examples:b}, containing a multi-line expression on Lines 5 and 6.}
		\label{fig:multiline:a}
  \end{subfigure}
  \hfill
  \begin{subfigure}[b]{0.45\textwidth}
		\begin{lstlisting}[language=C, firstnumber=5]
#if defined(A) && defined(B)

#else 
#endif
		\end{lstlisting}
		\caption{The same code with single-line expressions.\vspace{1.15em}}
		\label{fig:multiline:b}
  \end{subfigure}
  
  \caption{An \ifdef expression in file\,\code{Y.c}, (a) before and (b) after rewriting multi-line expressions into single-line fashion.}
  
\end{figure}

\end{frame}


\begin{frame}[fragile]
\frametitle{Rewriting of \ifdeff{ifdef} and \ifdeff{ifndef}}

\vfill
\ifdeff{ifdef E} \hfill $\xrightarrow{\hspace*{4em}}$ \hfill \ifdeff{if defined(E)}

\vfill

\ifdeff{ifndef F} \hfill $\xrightarrow{\hspace*{4em}}$ \hfill \ifdeff{if !defined(F)}

\vfill

\end{frame}


\begin{frame}[fragile]
\frametitle{Removal of Include Guards}


\begin{figure}[ht]
  \centering
  \begin{subfigure}[b]{0.4\textwidth}
	\begin{lstlisting}[language=C, firstnumber=8]
#ifndef Z_H
#define Z_H
#ifdef C
#elif defined(D)
#endif
#endif // Z_H
		\end{lstlisting}
		\caption{Include~guard in file\,\code{Z.h} (Lines 8, 9, and 13).}
		\label{fig:includeguard:a}
  \end{subfigure}
  \hfill
  \begin{subfigure}[b]{0.52\textwidth}
					\begin{lstlisting}[language=C, firstnumber=8]


#if defined(C)
#elif (!(defined(C))) && (defined(D))
#endif
@\phantom{\,}@
					\end{lstlisting}
		\caption{The example after removal of the include guard.}
		\label{fig:includeguard:b}
  \end{subfigure}
  
  \caption{The file\,\code{Z.h} (a) before and (b) after removal of the include guard.}
  
\end{figure}


\end{frame}


\begin{frame}[fragile]
\frametitle{Preparation: Before and After}


\begin{figure}[ht]
        \centering
        \small
        \begin{subfigure}[b]{0.4\textwidth}
					\begin{lstlisting}[language=C]
#ifdef A
  #ifdef B
  #endif
#endif
#ifdef A
#endif
					\end{lstlisting}
					\begin{lstlisting}[language=C, firstnumber=5]
#if  defined(A) \
     && defined(B)
#else 
#endif
					\end{lstlisting}
					\begin{lstlisting}[language=C, firstnumber=8]
#ifndef Z_H
#define Z_H
#ifdef C
#elif defined(D)
#endif
#endif // Z_H
					\end{lstlisting}
					\caption{\scriptsize Before preparation.}
        \end{subfigure}   
        \hfill
        \begin{subfigure}[b]{0.4\textwidth}
					\begin{lstlisting}[language=C]
#if defined(A)
  #if defined(B)
  #endif
#endif
#if defined(A)
#endif
					\end{lstlisting}
					\begin{lstlisting}[language=C, firstnumber=5]
#if  defined(A) && defined(B)

#else 
#endif
					\end{lstlisting}
					\begin{lstlisting}[language=C, firstnumber=8]


#if defined(C)
#elif defined(D)
#endif
@\phantom{a}@
					\end{lstlisting}
					\caption{\scriptsize After preparation.}
%					\label{fig:examples:a}
        \end{subfigure}         
        
        \label{fig:examples2_5}
\end{figure}

\end{frame}


\section[Processing *]{Processing of Expressions During File Analysis *}
\stepcounter{subsection}


\begin{frame}
\frametitle{Processing Steps}


\begin{itemize}

\item Look at each \ifdef expression.
\item Rewrite if nested, \ifdeff{elif}, or \ifdeff{else}  \textcolor{colorPassauAlert}{*}
	\begin{itemize}
	\item[\itshape nested] conjoin expressions from inner and outer \ifdefs
	\item[\itshape \ifdeff{elif}] conjoin expression with negation of all previous expressions
	\item[\itshape \ifdeff{else}] use conjunction of all previous expressions as negations
	\end{itemize}

\end{itemize}

\end{frame}



\begin{frame}[fragile]
\frametitle{Processing \ifdef Expressions \textcolor{colorPassauAlert}{*}}


\begin{figure}[ht]
        \centering
        \small
        \begin{subfigure}[b]{0.4\textwidth}
					\begin{lstlisting}[language=C]
#if defined(A)
  #if defined(B)
  #endif
#endif
#if defined(A)
#endif
					\end{lstlisting}
					\begin{lstlisting}[language=C, firstnumber=5]
#if  defined(A) && defined(B)

#else 
#endif
					\end{lstlisting}
					\begin{lstlisting}[language=C, firstnumber=8]


#if defined(C)
#elif defined(D)
#endif
@\phantom{a}@
					\end{lstlisting}
					\caption{\scriptsize Before processing.}
%					\label{fig:examples:a}
        \end{subfigure}
        \hfill
        \begin{subfigure}[b]{0.525\textwidth}
					\begin{lstlisting}[language=C]
#if defined(A)
  #if @\textbf{\textcolor{colorPassauAlert}{defined(A) \&\& }}@defined(B)
  #endif
#endif
#if defined(A)
#endif
					\end{lstlisting}
					\begin{lstlisting}[language=C, firstnumber=5]
#if  defined(A) && defined(B)

@\textbf{\textcolor{colorPassauAlert}{\#elif !(defined(A) \&\& defined(B))}}@
#endif
					\end{lstlisting}
					\begin{lstlisting}[language=C, firstnumber=8]


#if defined(C)
#elif @\textbf{\textcolor{colorPassauAlert}{(!(defined(C))) \&\& }}@(defined(D))
#endif
@\phantom{a}@
					\end{lstlisting}
					\caption{\scriptsize After processing.}
%					\label{fig:examples:a}
        \end{subfigure}   
        
        \label{fig:examples2}
\end{figure}

\end{frame}

  
  
\section[Global Pool *]{Global \ifdef Expression Pool *}
\stepcounter{subsection}


\begin{frame}
\frametitle{Global \ifdef Expression Pool}

\begin{itemize}

\item Local \ifdef Expression Pools \textcolor{colorPassauAlert}{*}
	\begin{itemize}
	\item Take each expression \textbf{only once} per file
	\item Comparison via string equality
	\end{itemize}

\item Construction of Global Expression Pool

\end{itemize}

\end{frame}


\begin{frame}[fragile]
\frametitle{Local \ifdef Expression Pools \textcolor{colorPassauAlert}{*}}



\begin{figure}[ht]
        \centering
        \small
        \begin{subfigure}[c]{0.52\textwidth}
					\begin{lstlisting}[language=C]
#if defined(A)
	#if defined(A) && defined(B)
	#endif
#endif
#if defined(A)
#endif
					\end{lstlisting}
				\end{subfigure}
				\hfill
        \begin{subfigure}[c]{0.43\textwidth}
					\begin{lstlisting}[language=C]
@\phantom{\,}@defined(A)
defined(A) && defined(B)
@\textbf{\textcolor{colorPassauAlert}{\sout{defined(A)}}}@
					\end{lstlisting}					
				\end{subfigure}	
\end{figure}%	
\begin{figure}[ht]
        \centering
        \small
        \begin{subfigure}[c]{0.52\textwidth}
					\begin{lstlisting}[language=C, firstnumber=5]
#if  defined(A) && defined(B)

#elif !(defined(A) && defined(B))
#endif
					\end{lstlisting}
				\end{subfigure}
				\hfill
        \begin{subfigure}[c]{0.43\textwidth}
					\begin{lstlisting}[language=C]
@\phantom{\,}@defined(A) && defined(B)
!(defined(A) && defined(B))
					\end{lstlisting}					
				\end{subfigure}	
\end{figure}%	
\begin{figure}[ht]
        \centering
        \small
        \begin{subfigure}[c]{0.52\textwidth}
					\begin{lstlisting}[language=C, firstnumber=8]


#if defined(C)
#elif (!(defined(C))) && (defined(D))
#endif
@\phantom{a}@
					\end{lstlisting}
				\end{subfigure}
				\hfill
        \begin{subfigure}[c]{0.43\textwidth}
					\begin{lstlisting}[language=C]
@\phantom{\,}@defined(C)
(!(defined(C))) && (defined(D))
					\end{lstlisting}					
				\end{subfigure}	
\end{figure}			


\end{frame}


\begin{frame}[fragile]
\frametitle{Construction of Global Expression Pool}

\begin{figure}

	\begin{subfigure}[c]{0.45\textwidth}

		\begin{lstlisting}[language=C]
@\phantom{\,}@defined(A)
defined(A) && defined(B)
		\end{lstlisting}

		\begin{lstlisting}[language=C]
@\phantom{\,}@defined(A) && defined(B)
!(defined(A) && defined(B))
		\end{lstlisting}		
	
		\begin{lstlisting}[language=C]
@\phantom{\,}@defined(C)
(!(defined(C))) && (defined(D))
		\end{lstlisting}
	
	\caption{\scriptsize Pools per file.}
	\end{subfigure}
	\hfill
	\begin{subfigure}[c]{0.45\textwidth}

		\begin{lstlisting}[language=C]
@\phantom{\,}@defined(A)
defined(A) && defined(B)
defined(A) && defined(B)
!(defined(A) && defined(B))
defined(C)
(!(defined(C))) && (defined(D))
		\end{lstlisting}
	
	\caption{\scriptsize Global expression pool.}
	\end{subfigure}


\end{figure}

\begin{block}{Note}

Scattering and tangling analyses are performed based on the global pool!

\end{block}

\end{frame}


\section{Summary}
\stepcounter{subsection}

\begin{frame}
\frametitle{Summary}

\begin{itemize}

\item Processing of \ifdef expressions and construction of local expression pools (one per file) affect scattering and tangling analyses that work per \ifdef.
\item Nesting analyses is not affected.

\end{itemize}

\end{frame}
  
\end{document}
